The MVR Package can be used to transfer water from a provider to a receiver.  Providers are extraction wells, streamflow routing reaches, lakes and other model features that can be conceptualized as having water available.  The list of packages that can provide water to the MVR Package are:

\begin{itemize}
  \item Well Package
  \item Drain Package
  \item River Package
  \item General-Head Boundary Package
  \item Multi-Aquifer Well Package
  \item Streamflow Routing Package
  \item Unsaturated Zone Flow Package
  \item Lake Package
\end{itemize}

Receivers are package features within the model that solve a continuity equation of inflows, outflows, and change in storage.  These features include multi-aquifer wells, streamflow routing reaches, lakes, and unsaturated zone flow cells.  The list of packages that can receive water is shorter than the provider list, because the WEL, DRN, RIV, and GHB Packages do not represent a continuity equation (boundary stages or elevations are specified by the user).  Therefore, the list of packages that can act as receivers are:

\begin{itemize}
  \item Multi-Aquifer Well Package
  \item Streamflow Routing Package
  \item Unsaturated Zone Flow Package
  \item Lake Package
\end{itemize}

\noindent The program will terminate with an error if the MVR is used with an unsupported package type.

The MVR Package is based on the calculation of available water that can be moved from one package feature to another.  The equations used to determine how much water can be transferred are as follows, where $Q_P$ is the flow rate that can be supported by the provider (the available flow rate), and $Q_R$ is the actual rate of water transferred to the receiver.

\begin{enumerate}
\item A FACTOR can be specified such that 

$Q_R = \alpha Q_P$

\noindent where $\alpha$ is the factor to convert the provider flow rate to the receiver flow rate.

\item An EXCESS rate can be specified by the user as $Q_S$ such that

\[
    Q_R = 
\begin{cases}
    Q_P - Q_S, & \text{if } Q_P > Q_S \\
    0,              & \text{otherwise}
\end{cases}
\]

\noindent In the EXCESS case, any water that exceeds the user specified rate is provided to the receiver.  No water is provided to the receiver if the available water is less than the user specified value.

\item A THRESHOLD rate can be specified for $Q_S$ such that

\[
    Q_R = 
\begin{cases}
    0, & \text{if } Q_S > Q_P \\
    Q_S,              & \text{otherwise}
\end{cases}
\]

\noindent In the THRESHOLD case, no flow is provided to the receiver until the available water exceeds the user specified $Q_S$ rate.  Once the available water exceeds the user specified rate, then the $Q_S$ rate is provided to the receiver.

\item An UPTO rate can be specified for $Q_S$ such that

\[
    Q_R = 
\begin{cases}
    Q_S, & \text{if } Q_P > Q_S \\
    Q_P,              & \text{otherwise}
\end{cases}
\]

\noindent In the UPTO case, all of the available water will be taken from the provider up to the $Q_S$ value specified by the user.  Once $Q_S$ is exceeded, the receiver will continue to get the $Q_S$ value specified by the user.
\end{enumerate}

\noindent In the MVR PERIOD block (as shown below), the user assigns the  equation to used for each individual entry by specifying FACTOR, EXCESS, THRESHOLD, or UPTO to the input variable \texttt{mvrtype}.

Input to the Water Mover (MVR) Package is read from the file that has type ``MVR6'' in the Name File.  Only one MVR Package can be used per GWF Model.  All single valued variables are free format.

\vspace{5mm}
\subsubsection{Structure of Blocks}
\vspace{5mm}

\noindent \textit{FOR EACH SIMULATION}
\lstinputlisting[style=blockdefinition]{./mf6ivar/tex/gwf-mvr-options.dat}
\lstinputlisting[style=blockdefinition]{./mf6ivar/tex/gwf-mvr-dimensions.dat}
\lstinputlisting[style=blockdefinition]{./mf6ivar/tex/gwf-mvr-packages.dat}
\vspace{5mm}
\noindent \textit{FOR ANY STRESS PERIOD}
\lstinputlisting[style=blockdefinition]{./mf6ivar/tex/gwf-mvr-period.dat}

\vspace{5mm}
\subsubsection{Explanation of Variables}
\begin{description}
% DO NOT MODIFY THIS FILE DIRECTLY.  IT IS CREATED BY mf6ivar.py 

\item \texttt{PRINT\_INPUT}---keyword to indicate that the list of MVR information will be written to the listing file immediately after it is read.

\item \texttt{PRINT\_FLOWS}---keyword to indicate that the list of MVR flow rates will be printed to the listing file for every stress period time step in which ``BUDGET PRINT'' is specified in Output Control.  If there is no Output Control option and \texttt{PRINT\_FLOWS} is specified, then flow rates are printed for the last time step of each stress period.

\item \texttt{MODELNAMES}---keyword to indicate that all package names will be preceded by the model name for the package.  Model names are required when the Mover Package is used with a GWF-GWF Exchange.  The MODELNAME keyword should not be used for a Mover Package that is for a single GWF Model.

\item \texttt{BUDGET}---keyword to specify that record corresponds to the budget.

\item \texttt{FILEOUT}---keyword to specify that an output filename is expected next.

\item \texttt{budgetfile}---name of the output file to write budget information.

\item \texttt{maxmvr}---integer value specifying the maximum number of water mover entries that will specified for any stress period.

\item \texttt{maxpackages}---integer value specifying the number of unique packages that are included in this water mover input file.

\item \texttt{mname}---name of model containing the package.

\item \texttt{pname}---is the name of a package that may be included in a subsequent stress period block.

\item \texttt{iper}---integer value specifying the starting stress period number for which the data specified in the PERIOD block apply.  \texttt{iper} must be less than \texttt{nper} in the TDIS Package and greater than zero.  The \texttt{iper} value assigned to a stress period block must be greater than the \texttt{iper} value assigned for the previous block.

\item \texttt{mname1}---name of model containing the package, \texttt{pname1}.

\item \texttt{pname1}---is the package name for the provider.  The package \texttt{pname1} must be designated to provide water through the MVR Package by specifying the keyword ``MOVER'' in its OPTIONS block.

\item \texttt{id1}---is the identifier for the provider.  This is the well number, reach number, lake number, etc.

\item \texttt{mname2}---name of model containing the package, \texttt{pname2}.

\item \texttt{pname2}---is the package name for the receiver.  The package \texttt{pname2} must be designated to receive water from the MVR Package by specifying the keyword ``MOVER'' in its OPTIONS block.

\item \texttt{id2}---is the identifier for the receiver.  This is the well number, reach number, lake number, etc.

\item \texttt{mvrtype}---is the character string signifying the method for determining how much water will be moved.  Supported values are ``FACTOR'' ``EXCESS'' ``THRESHOLD'' and ``UPTO''.  These four options determine how the receiver flow rate, $Q_R$, is calculated.  These options are based the options available in the SFR2 Package for diverting stream flow.

\item \texttt{value}---is the value to be used in the equation for calculating the amount of water to move.  For the ``FACTOR'' option, \texttt{value} is the $\alpha$ factor.  For the remaining options, \texttt{value} is the specified flow rate, $Q_S$.



\end{description}

\vspace{5mm}
\subsubsection{Example Input File}
\lstinputlisting[style=inputfile]{./mf6ivar/examples/gwf-mvr-example.dat}
