% DO NOT MODIFY THIS FILE DIRECTLY.  IT IS CREATED BY mf6ivar.py 

\item \textbf{Block: OPTIONS}

\begin{description}
\item \texttt{SAVE\_FLOWS}---keyword to indicate that cell-by-cell flow terms will be written to the file specified with ``BUDGET SAVE FILE'' in Output Control.

\item \texttt{STORAGECOEFFICIENT}---keyword to indicate that the \texttt{ss} array is read as storage coefficient rather than specific storage.

\end{description}
\item \textbf{Block: GRIDDATA}

\begin{description}
\item \texttt{iconvert}---is a flag for each cell that specifies whether or not a cell is convertible for the storage calculation.  0 indicates confined storage is used. $>$0 indicates confined storage is used when head is above cell top and unconfined storage is used when head is below cell top.  A mixed formulation is when when a cell converts from confined to unconfined (or vice versa) during a single time step.

\item \texttt{ss}---is specific storage (or the storage coefficient if STORAGECOEFFICIENT is specified as an option).

\item \texttt{sy}---is specific yield.

\end{description}
\item \textbf{Block: PERIOD}

\begin{description}
\item \texttt{iper}---integer value specifying the starting stress period number for which the data specified in the PERIOD block apply.  \texttt{iper} must be less than or equal to \texttt{nper} in the TDIS Package and greater than zero.  The \texttt{iper} value assigned to a stress period block must be greater than the \texttt{iper} value assigned for the previous PERIOD block.  The information specified in the PERIOD block will continue to apply for all subsequent stress periods, unless the program encounters another PERIOD block.

\item \texttt{STEADY-STATE}---keyword to indicate that stress-period \texttt{iper} is steady-state. Steady-state conditions will apply until the \texttt{TRANSIENT} keyword is specified in a subsequent \texttt{BEGIN PERIOD} block.

\item \texttt{TRANSIENT}---keyword to indicate that stress-period \texttt{iper} is transient. Transient conditions will apply until the \texttt{STEADY-STATE} keyword is specified in a subsequent \texttt{BEGIN PERIOD} block.

\end{description}

