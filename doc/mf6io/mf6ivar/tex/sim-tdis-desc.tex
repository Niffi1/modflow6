% DO NOT MODIFY THIS FILE DIRECTLY.  IT IS CREATED BY mf6ivar.py 

\item \textbf{Block: OPTIONS}

\begin{description}
\item \texttt{time\_units}---is the time units of the simulation.  This is a text string that is used as a label within model output files.  Values for time\_units may be ``unknown'',  ``seconds'', ``minutes'', ``hours'', ``days'', or ``years''.  The default time unit is ``unknown''.

\item \texttt{start\_date\_time}---is the starting date and time of the simulation.  This is a text string that is used as a label within the simulation list file.  The value has no affect on the simulation.  The recommended format for the starting date and time is described at https://www.w3.org/TR/NOTE-datetime.

\end{description}
\item \textbf{Block: DIMENSIONS}

\begin{description}
\item \texttt{nper}---is the number of stress periods for the simulation.

\end{description}
\item \textbf{Block: PERIODDATA}

\begin{description}
\item \texttt{perlen}---is the length of a stress period.

\item \texttt{nstp}---is the number of time steps in a stress period.

\item \texttt{tsmult}---is the multiplier for the length of successive time steps. The length of a time step is calculated by multiplying the length of the previous time step by TSMULT. The length of the first time step, $\Delta t_1$, is related to PERLEN, NSTP, and TSMULT by the relation $\Delta t_1= perlen \frac{tsmult - 1}{tsmult^{nstp}-1}$.

\end{description}

