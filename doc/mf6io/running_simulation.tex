\mf is run from the command line by entering the name of the \mf executable program.  If the run is successful, it will conclude with a statement about normal termination.

{\small
\begin{lstlisting}[style=modeloutput]
                                   MODFLOW 6
                U.S. GEOLOGICAL SURVEY MODULAR HYDROLOGIC MODEL
                        VERSION mf6.0.0 August 11, 2017

    MODFLOW 6 compiled Aug 07 2017 10:52:44 with IFORT compiler (ver. 17.00)

This software is preliminary or provisional and is subject to revision.
It is being provided to meet the need for timely best science. The
software has not received final approval by the U.S. Geological Survey
(USGS). No warranty, expressed or implied, is made by the USGS or the
U.S. Government as to the functionality of the software and related
material nor shall the fact of release constitute any such warranty. The
software is provided on the condition that neither the USGS nor the U.S.
Government shall be held liable for any damages resulting from the
authorized or unauthorized use of the software.

 Run start date and time (yyyy/mm/dd hh:mm:ss): 2017/08/07 14:30:04

 Writing simulation list file: mfsim.lst
 Using Simulation name file: mfsim.nam
 Solving:  Stress period:     1    Time step:     1
 Run end date and time (yyyy/mm/dd hh:mm:ss): 2017/08/07 14:30:04
 Elapsed run time:  0.187 Seconds

 Normal termination of simulation.
\end{lstlisting}
}

\noindent \mf requires that a simulation name file (described in a subsequent section titled ``Simulation Name File'') be present in the working directory.  This simulation name file must be named ``mfsim.nam''.  If the mfsim.nam file is not located in the present working directory, then \mf will terminate with the following error.  

{\small
\begin{lstlisting}[style=modeloutput]
                                   MODFLOW 6
                U.S. GEOLOGICAL SURVEY MODULAR HYDROLOGIC MODEL
                        VERSION mf6.0.0 August 11, 2017

    MODFLOW 6 compiled Aug 07 2017 10:52:44 with IFORT compiler (ver. 17.00)

This software is preliminary or provisional and is subject to revision.
It is being provided to meet the need for timely best science. The
software has not received final approval by the U.S. Geological Survey
(USGS). No warranty, expressed or implied, is made by the USGS or the
U.S. Government as to the functionality of the software and related
material nor shall the fact of release constitute any such warranty. The
software is provided on the condition that neither the USGS nor the U.S.
Government shall be held liable for any damages resulting from the
authorized or unauthorized use of the software.

 Run start date and time (yyyy/mm/dd hh:mm:ss): 2017/08/07 14:31:28

 Writing simulation list file: mfsim.lst

ERROR REPORT:

 *** ERROR OPENING FILE "mfsim.nam" ON UNIT 1001
       SPECIFIED FILE STATUS: OLD
       SPECIFIED FILE FORMAT: FORMATTED
       SPECIFIED FILE ACCESS: SEQUENTIAL
       SPECIFIED FILE ACTION: READ
         IOSTAT ERROR NUMBER: 29
  -- STOP EXECUTION (openfile)
 Stopping due to error(s)
\end{lstlisting}
}

During execution \mf creates a simulation output file, called a listing file, with the name ``mfsim.lst''.  This file contains general simulation information, including information about exchanges between models, timing, and solver progress.  Separate listing files are also written for each individual model.  These listing files contains the details for the specific models.

In the event that \mf encounters an error, the error message is written to the command line window as well as to the simulation listing file.  The error message will also contain the name of the file that was being read when the error occurred, if possible.  This information can be used to diagnose potential causes of the error.  
